\RequirePackage{silence}
\WarningFilter{remreset}{The remreset package}
\documentclass[11pt]{book}
\usepackage{amsmath,amsthm,amssymb}
\usepackage[hidelinks]{hyperref}
\usepackage[margin=1in]{geometry}
\usepackage{standalone}
\usepackage{fancyhdr}
\usepackage{tikz}
\usepackage{mathtools}
\usepackage{thmtools}
\usepackage{setspace}
\usepackage{enumitem}
\usepackage{xcolor}
\hypersetup{
    colorlinks,
    linkcolor={red!50!black},
    citecolor={blue!50!black},
    urlcolor={blue!80!black}
}
%\usepackage{tcolorbox}

%%%%%%%%%%%%% HEADERS %%%%%%%%%%%%%%%
\pagestyle{fancy}
\fancyhf{}
\rhead{\thepage}
\lhead{\leftmark}
\headheight=13.6pt

%%%%%%%%%%%%% ENVIRONMENTS %%%%%%%%%%%%%%%
\declaretheorem[numberwithin=section, style=definition]{definition}
\declaretheorem[sharenumber=definition, style=definition]{theorem}
\declaretheorem[sharenumber=definition, style=definition]{exercise}
\declaretheorem[sharenumber=definition, style=definition]{lemma}
\declaretheorem[sharenumber=definition, style=definition]{proposition}
\declaretheorem[sharenumber=definition, style=definition]{corollary}
\declaretheorem[sharenumber=definition, style=definition]{remark}
\declaretheorem[sharenumber=definition, style=definition]{axiom}
\declaretheorem[sharenumber=definition, style=definition]{example}
\declaretheorem[sharenumber=definition, style=definition]{result}
\declaretheorem[sharenumber=definition, style=definition]{problem}
\declaretheorem[sharenumber=definition, style=definition]{conjecture}
\declaretheorem[sharenumber=definition, style=definition]{algorithm}
\declaretheorem[sharenumber=definition, style=definition]{heuristic}
\declaretheorem[sharenumber=definition, style=definition]{motivation}
\declaretheorem[sharenumber=definition, style=definition]{intuition}
\declaretheorem[sharenumber=definition, style=definition]{anecdote}
\newenvironment{solution}{\begin{proof}[Solution]}{\end{proof}}
\newenvironment{abc}{\begin{enumerate}[label=(\alph*)]}{\end{enumerate}}
\newcommand{\env}[2]{\begin{#1}#2\end{#1}}

%%%%%%%%%%%%% SHORTCUTS %%%%%%%%%%%%%%%
\DeclareMathOperator{\N}{\mathbf{N}}
\DeclareMathOperator{\Z}{\mathbf{Z}}
\DeclareMathOperator{\R}{\mathbf{R}}
\DeclareMathOperator{\F}{\mathbf{F}}
\DeclareMathOperator{\Q}{\mathbf{Q}}
\DeclareMathOperator{\GL}{\mathrm{GL}}
\DeclareMathOperator{\SL}{\mathrm{SL}}
\DeclareMathOperator{\lcm}{\operatorname{lcm}}
\DeclareMathOperator{\ord}{\operatorname{ord}}
\DeclareMathOperator{\sgn}{\operatorname{sgn}}
\DeclareMathOperator{\Aut}{\operatorname{Aut}}
\DeclareMathOperator{\Inn}{\operatorname{Inn}}
\DeclareMathOperator{\End}{\operatorname{End}}
\DeclareMathOperator{\stab}{\operatorname{stab}}
\DeclareMathOperator{\orb}{\operatorname{orb}}
%\DeclareMathOperator{\ker}{\operatorname{ker}}
\DeclareMathOperator{\im}{\operatorname{im}}
\DeclareMathOperator{\IM}{\operatorname{Im}}
\DeclareMathOperator{\RE}{\operatorname{Re}}
\DeclareMathOperator{\Span}{\operatorname{Span}}
\DeclareMathOperator{\Spec}{\operatorname{Spec}}
\DeclareMathOperator{\Char}{\operatorname{char}}
\DeclareMathOperator{\Rank}{\operatorname{rank}}
\DeclareMathOperator{\proj}{\operatorname{proj}}
\DeclareMathOperator{\card}{\operatorname{card}}
\DeclareMathOperator{\normal}{\trianglelefteq}
\DeclareMathOperator{\vep}{\varepsilon}
\DeclareMathOperator{\vphi}{\varphi}

\begin{document}
\title{\Huge \textbf{Introduction to Algebra and Analysis}}
\author{Yunhai Xiang}
\date{\today}
\setcounter{tocdepth}{1}
\maketitle
\doublespacing
\tableofcontents
\singlespacing
\chapter{Foundations of Mathematics}
\section{Motivation}
In the 1930s, a group of mainly French mathematicians under the pseudonym \textsl{Bourbaki} decided to reformulate mathematics on an extremely abstract, formal and rigorous manner. They undertook arduous efforts and  publish the treatise called \textsl{\'El\'ements de math\'ematique}, the first book of which, \textsl{Th\'eorie des ensembles} or \textsl{Theory of Sets} in English, is the foundation of all mathematics.

One of our objectives for axiomatic set theory is to avoid paradoxes such as the \textbf{Russell paradox}, which comes from the very naive intuition that if formula $F$ about $x$ does not use $w$ then
\[\exists w\,\forall x\,[(x\in w)\leftrightarrow F]\]
and this obviously creates a paradox, since if we take $F\equiv (x\not\in x)$, then $(w\in w)\leftrightarrow (w\not\in w)$. As an analogy, consider a barber who shaves all those, and those only, who do not shave themselves, then does the barber shave himself? 

\newpage 
\section{Formal Languages}
In this section, we will introduce the most basic concept in the theory of sets: \textbf{formal languages}, specifically, the \textbf{propositional language} and the \textbf{first order language}.
A string in propositional language consists of the following symbols

\begin{enumerate}[label=\roman*.]
	\item Proposition Symbols: $A,B,C,D,\dots$
	\item Negation Symbols: $\neg$ (not)
	\item Binary Logic Symbols: $\land$ (and), $\lor$ (or), $\rightarrow$ (implies), $\leftrightarrow$ (if and only if)
	\item Auxiliary Symbols: $(,),[,]$
\end{enumerate}
and a \textbf{formula} in propositional language is a string that follows the following rules
\begin{enumerate}
	\item If $A$ is a proposition symbol, $A$ is a formula
	\item If $F$ is a formula, then $\neg F$ is a formula
	\item If $F,G$ are formulas, then $F\land G,F\lor G,F\rightarrow G,F\leftrightarrow G$ are formulas
	\item The auxiliary brackets $(,)$ are often used to clearify the structure of the formula.
\end{enumerate}


Let $F$ be a formula with proposition symbols $A_1,\dots,A_n$, then each proposition symbol can be assigned a \textbf{boolean} value: true or false, represented by $1$ and $0$ respectively. Therefore, for a specific assignment of $A_1,\dots,A_n$ we can obtain a \textbf{valuation} of $F$ through reduction by the reduction rules specified in the following tables
\[
\begin{array}{|c|c|}
\hline A & \neg A \\ 
\hline 1 & 0 \\ 
\hline 0 & 1 \\ 
\hline
\end{array}
\hspace{3mm}
\begin{array}{|c|c|c|}
\hline A & B & A\land B \\
\hline 1 & 1 & 1 \\
\hline 1 & 0 & 0 \\
\hline 0 & 1 & 0 \\
\hline 0 & 0 & 0 \\
\hline
\end{array}
\hspace{3mm}
\begin{array}{|c|c|c|}
\hline A & B & A\lor B \\
\hline 1 & 1 & 1 \\
\hline 1 & 0 & 1 \\
\hline 0 & 1 & 1 \\
\hline 0 & 0 & 0 \\
\hline
\end{array}
\hspace{3mm}
\begin{array}{|c|c|c|}
\hline A & B & A\rightarrow B \\
\hline 1 & 1 & 1 \\
\hline 1 & 0 & 0 \\
\hline 0 & 1 & 1 \\
\hline 0 & 0 & 1 \\
\hline
\end{array}
\hspace{3mm}
\begin{array}{|c|c|c|}
\hline A & B & A\leftrightarrow B \\
\hline 1 & 1 & 1 \\
\hline 1 & 0 & 0 \\
\hline 0 & 1 & 0 \\
\hline 0 & 0 & 1 \\
\hline
\end{array}
\]
\env{example}{
	\label{example:form}
	Consider the following formulas
	\begin{enumerate}[label=(\alph*)]
		\item $((A\lor B)\land A)\rightarrow \lnot B$
		\item $((A\rightarrow B)\rightarrow A)\rightarrow A$
		\item $(\lnot A\rightarrow\lnot B)\leftrightarrow \lnot(A\rightarrow\lnot B)$
	\end{enumerate}
	A valuation of (a) by assigning both $A$ and $B$ as true would be
	\[
	\begin{aligned}
	&((A\lor B)\land A)\rightarrow \lnot B\\
	\Rightarrow\,&((1\lor 1)\land 1)\rightarrow \lnot 1\\
	\Rightarrow\,&((1\lor 1)\land 1)\rightarrow 0\\
	\Rightarrow\,&(1\land 1)\rightarrow 0\\
	\Rightarrow\,&1\rightarrow 0\\
	\Rightarrow\,& 0\\
	\end{aligned}
	\]
	\iffalse
	\noindent As another example, a valuation of (b) by assigning $A$ as true and $B$ as false is 
	\[
	\begin{aligned}
	&((A\rightarrow B)\rightarrow A)\rightarrow A\\
	\Rightarrow\,&((1\rightarrow 0)\rightarrow 1)\rightarrow 1\\
	\Rightarrow\,&(0\rightarrow 1)\rightarrow 1\\
	\Rightarrow\,& 1\rightarrow 1\\
	\Rightarrow\,& 1\\
	\end{aligned}
	\]
	and in fact, (b) will always be evaluated to be true no matter what boolean values $A,B$ is assigned (verify this!). \fi 
	As an exercise, obtain the valuations of (a),(b) and (c) of all assignments for $A,B$.
}
Formulas such as (b) in Example \ref{example:form} that are always valued to be true no matter what boolean values its variables is assigned is called a \textbf{tautology}. We write $\vDash F$ to denote that $F$ is a tautology.

\begin{example}Here are some examples of tautologies,
\env{enumerate}{
	\item (Law of Identity) $\vDash A\leftrightarrow A$
	\item (Law of Double Negativity) $\vDash \lnot\lnot A\leftrightarrow A$
	\item (Law of Excluded Middle) $\vDash A\lor \lnot A$
	\item (Law of Idempotence) $\vDash A\leftrightarrow (A\land A)$
	\item (Law of Idempotence) $\vDash A\leftrightarrow (A\lor A)$
	\item (Law of Commutativity) $\vDash (A\lor B)\leftrightarrow (B\lor A)$
	\item (Law of Commutativity) $\vDash (A\land B)\leftrightarrow (B\land A)$
	\item (Law of Associativity) $\vDash ((A\land B)\land C)\leftrightarrow (A\land (B\land C))$
	\item (Law of Associativity) $\vDash ((A\lor B)\lor C)\leftrightarrow (A\lor (B\lor C))$
	\item (Law of Distributivity) $\vDash ((A\land B)\lor C)\leftrightarrow ((A\lor C)\land (B\lor C))$
	\item (Law of Distributivity) $\vDash ((A\lor B)\land C)\leftrightarrow ((A\land C)\lor (B\land C))$
	\item (Law of Absorption) $\vDash (A\land (A\lor B))\leftrightarrow A$
	\item (Law of Absorption) $\vDash (A\lor (A\land B))\leftrightarrow A$
	\item (De Morgan's Law) $\vDash \lnot(A\lor B)\leftrightarrow (\lnot A \land \lnot B)$
	\item (De Morgan's Law) $\vDash \lnot(A\land B)\leftrightarrow (\lnot A \lor \lnot B)$
	\item (Law of Implication) $\vDash (A\rightarrow B)\leftrightarrow (\neg A\lor B)$
	\item (Law of Implication) $\vDash \neg (A\rightarrow B)\leftrightarrow (A\land \neg B)$
	\item (Law of If and Only If) $\vDash (A\leftrightarrow B)\leftrightarrow ((A\rightarrow B)\land (B\rightarrow A))$
	\item (Law of Tautology) $\vDash (A\land (B\lor\neg B))\leftrightarrow A$
	\item (Law of Tautology) $\vDash (A\lor (B\lor\neg B))\leftrightarrow (B\lor\neg B)$
	\item (Law of Contradiction) $\vDash (A\land (B\land\neg B))\leftrightarrow (B\land\neg B)$
	\item (Law of Contradiction) $\vDash (A\lor (B\land\neg B))\leftrightarrow A$
	\item (Law of Contrapositive) $\vDash (A\rightarrow B)\leftrightarrow(\lnot B\rightarrow \lnot A)$
	\item (Peirce's Law) $\vDash ((A\rightarrow B)\rightarrow A)\rightarrow A$
	\item (Proof by Contradiction) $\vDash (\lnot A\rightarrow (B\land \lnot B))\rightarrow A$
	
	\item (Principle of Syllogism) $\vDash ((A\rightarrow B)\land (B\rightarrow C))\rightarrow (A\rightarrow C)$
}
\end{example}

First order language can be seen as a extension to propositional language. It is much more complicated as it introduces quantification of variables, which means that a variable in first order language is either a \textbf{free variable} or a \textbf{bound variable}. First order language is a major topic of this chapter, and it will be the primary formal language we will use in this book. We will also formulate the \textbf{ZFC axioms} in first order language, which you will see in the next section. A string in first order language consists of the following symbols,
\begin{enumerate}[label=\roman*.]
	\item Variable Symbols: $a,b,c,d,\dots$
	\item Predicate Symbols: $\in$ (in), $=$ (equals) 
	\item Quantifier Symbols: $\forall$ (forall), $\exists$ (exists)
	\item Negation Symbols: $\neg$ (not)
	\item Binary Logic Symbols: $\land$ (and), $\lor$ (or), $\rightarrow$ (implies), $\leftrightarrow$ (if and only if)
	\item Class Builder Symbols: $\{,|,\}$
	\item Auxiliary Symbols: $(,),[,]$
\end{enumerate}
A formula in first order language is a string that follows the following rules
\begin{enumerate}
	\item If $a,b$ are variables then $a\circ b$ is a formulas where $\circ$ is one of the predicate symbols $\in$ or $=$. The free variables of $a\circ b$ are $a$ and $b$, and $a\circ b$ does not have any bound variables.

	\item If $F$ is a formula then so is $\neg F$. The free variables of $\neg F$ is the same as the free variables of $F$, and the bound variables of $\neg F$ is the same as the bound variables of $F$.

	\item If $F,G$ are formulas such that any free variable in $F$ is not a bound variable in $G$, and any bound variable in $F$ is not a free variable in $G$, then $F\circ G$ is a formula where $\circ$ is one of the binary logic symbols $\land$, $\lor$, $\rightarrow$, or $\leftrightarrow$. The free variables of $F\circ G$ are the free variables of $F$ and $G$, and the bound variables $F\circ G$ are the bound variables of $F$ and $G$.

	\item If $a$ is a free variable of the formula $F$ then $\forall a\, F$ and $\exists a\, F$ are formulas. The free variables $\forall a\, F$ are the free variables of $F$ except $a$, and the bound variables of $\forall a\, F$ are $a$ and the bound variables of $F$. The same applies to $\exists a\, F$. 
	
	\item A term is a string that follows the following ruless
		\env{enumerate}{
			\item If $a$ is a variable symbol then $a$ is a term with one free variable $a$ and no bound variables.
			\item Let $a_1,\dots,a_n$ where $n\ge 1$ be variables symbols distinct from $x_1,\dots,x_m$ where $m\ge 0$. If $\phi$ is a term such that $a_1,\dots,a_n$ are its only free variables and none of $x_1,\dots,x_m$ is its bound variable, and $\varphi$ is a formula with $a_1,\dots,a_n,x_1,\dots,x_m$ as its only free variables, then $\{\phi\mid\varphi\}$ is a term. The free variables of $\{\phi\mid\varphi\}$ are $x_1,\dots,x_m$, and the bound variables of $\{\phi\mid\varphi\}$ are the bound variables of $\varphi$ and $\phi$ and also the variables $a_1,\dots,a_n$.
		}
		 If $s,t$ are terms such that any free variable in $s$ is not a bound variable in $t$, and any bound variable in $s$ is not a free variable in $t$, then $s\circ t$ is a formula where $\circ$ is one of the predicate symbols $\in$ or $=$. The free variables of $s\circ t$ are the free variables of the terms $s$ and $t$, and the bound variables of $s\circ t$ are the bound variables of the terms $s$ and $t$.
	\item The auxiliary brackets $(,)$ and $[,]$ are often used to clearify the structure of the formula.
\end{enumerate}
\env{example}{
	Consider the following strings
	\env{abc}{
		\item $\forall x\,\exists y \, [\neg ((x=y)\rightarrow (z\in y))\lor \exists a\, (z\in a \leftrightarrow z\in z)]$
		\item  $\forall x\,[(\forall z \,[(z\in a)\rightarrow (z\in z)])\leftrightarrow (\exists x\, (x\in y))]$
		\item $\exists a \,[\forall b\,[b\in \{x\mid (a\in x)\land (x\in c)\}]\leftrightarrow \forall b\,[(a\in b)\lor (b=c)]]$
		\item $(x\in y)\rightarrow \forall a\, [(\forall b\,[(b\in a)\rightarrow (b=c)])\land (\forall z\, (z\in b))]$
		\item $\{\{x\mid x=a\lor x=b\}\mid x\in y\land b\in y\}=\{\{g\mid g\in i\lor g\in j\lor g\in k \}\mid \exists l [l\in \{x\mid x\in i\rightarrow x\in j\}]\}$
	}
	and we can easily verify that string (a) is a valid formula with free variable $z$ and bound variables $x,y$ and $a$. The string (b), however, is not a valid formula, since $\exists x$ appeared within $\forall x$ which violates rule 4. As an exercise, check whether the strings (c),(d) and (e) are valid formulas, and if so, what are its free and bound variables? If not, which rule does it violate?
}
The notion of free and bound variables extends beyond formal language and can be applied to expressions you are probably more familiar with, for example, in the expression
\[\lim _{n \rightarrow \infty}\left[\left(\prod_{k=1}^{n} \frac{2 k}{2 k-1}\right) \int_{-1}^{\infty} \frac{(\cos x)^{2 n}}{2^{x}}\,\mathrm d x\right]=\frac{\pi 2^\pi}{2^\pi-1}\]
the variables $n,k,x$ are all bound variables and there are no free variables, hence it is a proposition\footnote{This is a problem taken from the 2013 Stanford Math Tournament. See appendix for proof.}.\vspace{-3mm}
\env{exercise}{
	Let $\{\cdot\}$ be the fractional part function, and consider the expression
	\[\int_{0}^{1} \cdots \int_{0}^{1}\left\{\frac{1}{\prod_{n=1}^{k} x_{n}}\right\} \,\mathrm d x_{1} \cdots\mathrm d x_{k}=1-\sum_{n=0}^{k-1} \left[\frac{1}{n!}\lim _{m \rightarrow \infty}\left[\sum_{\ell=1}^{m} \frac{(\ln \ell)^{n}}{\ell}-\frac{(\ln m)^{n+1}}{n+1}\right]\right]\]
	what are the free variables and what are the bound variables? Try proving this as a challenge.
}

\begin{remark}To make things easier for us, we introduce the following terminologies
\env{enumerate}{
	\item A formula with $x_1,\dots,x_n$ as is its only free variables is a \textbf{property} of $x_1,\dots,x_n$,
	\item A term with $x_1,\dots,x_n$ as its only free variables is a \textbf{mapping} of $x_1,\dots,x_n$,
	\item A formula without free variables is a \textbf{proposition},
	\item A term without free variables is called a \textbf{class},
	\item If a variable is neither free nor bound in a formula, it is \textbf{unused} in that formula,
	\item If a variable is neither free nor bound in a term, it is \textbf{unused} in that term,
	\item Let $a_1,\dots,a_n$ where $n\ge 1$ be variables symbols distinct from $x_1,\dots,x_m$ where $m\ge 0$. If $\phi$ is a mapping of $a_1,\dots,a_n$ and none of $x_1,\dots,x_m$ is its bound variable, and $\varphi$ is a property of $a_1,\dots,a_n,x_1,\dots,x_m$, we say that the string $\{\phi\mid\varphi\}$ \textbf{ranges over} $a_1,\dots,a_n$.
	\item If $F$ is a string, $x$ is a symbol and $s$ is another string, we use $[F][x\mapsto s]$ represents the string obtained by replacing every occurence of $x$ in $F$ by $s$.
	\iffalse
	\item If $F$ is a tautology in propositional language with proposition variables $A_1,\dots,A_n$ and $B_1,\dots,B_n$ are formulas in first order language, then we say that $[F][A_1\mapsto B_1]\cdots [A_n\mapsto B_n]$ is a \textbf{tautology} in first order language. If $G$ is a proposition in first order language, we write $\vDash G$ to indicate that $G$ is a tautology in first order language.\fi
}
\end{remark}
\iffalse
\noindent Let $s,t$ be terms, $F,G,H$ be formulas, and $x,y$ be variables
\env{enumerate}{
	\item $F\equiv G$ if $\vDash F\leftrightarrow G$
	\item $F\equiv H$ if $F\equiv G$ and $G\equiv H$
	\item $F\equiv G$ if $G\equiv F$
	\item $\forall x\,\forall y\, F\equiv \forall y\,\forall x\, F$ if $x,y$ are not bound in $F$
	\item $\exists x\,\exists y\, F\equiv \exists y\,\exists x\, F$ if $x,y$ are not bound in $F$
	\item $\neg \forall x\, F\equiv \exists x\,\neg F$ if $x$ is not bound in $F$
	\item $\neg \exists x\, F\equiv \forall x\,\neg F$ if $x$ is not bound in $F$
	\item $\forall x\,(F\land G)\equiv (\forall  x\, F)\land (\forall x\, G)$ if $x$ is not bound in $F$ and $G$
	\item $\exists x\,(F\lor G)\equiv (\exists  x\, F)\lor (\exists x\, G)$ if $x$ is not bound in $F$ and $G$
	\item $F\equiv \forall x\, F$ if $x$ is unused in $F$
	\item $F\equiv \exists x\, F$ if $x$ is unused in $F$
	\item $\forall x\,F\equiv \forall y\, [F][x\mapsto y]$ if $x$ is not bound and $y$ is unused in $F$.
	\item $\exists x\,F\equiv \exists y\, [F][x\mapsto y]$ if $x$ is not bound and $y$ is unused in $F$.
	\item 

}
Prove that
\env{enumerate}{
	\item $F\equiv F$
	\item $F\equiv \neg\neg F$
	\item $F\land F\equiv F$
	\item $F\lor F\equiv F$
	\item $F\land G\equiv G\land F$
	\item $F\lor G\equiv G\lor F$
	\item $F\land (G\land H)\equiv (F\land G)\land H$
	\item $F\lor (G\lor H)\equiv (F\lor G)\lor H$
	\item $\neg (F\land G)\equiv \neg F\lor \neg  G$
	\item $\neg (F\lor G)\equiv \neg F\land \neg  G$
	\item $F\land (F\lor G)\equiv F$
	\item $F\lor (F\land G)\equiv F$
	\item $F\land (G\lor \neg G)\equiv F$
	\item $F\lor (G\lor \neg G)\equiv G\lor \neg G$
	\item $F\land (G\land \neg G)\equiv G\land \neg G$
	\item $F\lor (G\land \neg G)\equiv F$
	\item $F\rightarrow G\equiv \neg G\rightarrow \neg F\equiv \neg F\lor G$
	\item $\neg (F\rightarrow G)\equiv F\land \neg G$
	\item $F\leftrightarrow G\equiv (F\land G)\lor (\neg F\land \neg G)\equiv (\neg F\lor G)\land (F\lor \neg G)\equiv (F\rightarrow G)\land (G\rightarrow F)$
	\item $F\land (G\lor H)\equiv (F\land G)\lor (F\land H)$
	\item $F\lor (G\land H)\equiv (F\lor G)\land (F\lor H)$
	\item $(s=t)\equiv (t=s)$
	
	\item $s\in\{\phi\mid \varphi\}\equiv s\in\{y\mid \exists a_1\,\cdots\exists a_n\,(y=\phi)\land \varphi\}$ if $y$ is unused in $\phi,\varphi$ and $s$.
	\item $s\in\{\phi\mid \varphi\}\equiv \exists a_1\,\cdots\exists a_n\,(s=\phi)\land \varphi$ if variables of $s$ are unused in $\phi,\varphi$ and $s$.
	\item $\{\phi\mid \varphi\}\in s\equiv \{y\mid \exists a_1\,\cdots\exists a_n\,(y=\phi)\land \varphi\}\in s$ if $y$ is unused in $\phi,\varphi$ and $s$.
	\item $\{\phi\mid \varphi\}\in s\equiv \exists y\,[y\in s\land y=\{\phi\mid\varphi\}]$ if $y$ is unused in $\phi,\varphi$ and $s$.
}
\fi
As the formulas we are considering gets longer and more complicated, we need to introduce the notion of \textbf{definitions}. Definitions are abbriviations of terms and formulas. There are two types of definitions, \textbf{extensional definitions} and \textbf{intensional definitions}. An extensional definition is an abbriviation of a mapping by introducing a new notation or terminology. To declare an extensional definition, we either write the string
\[[a_1,\dots,a_n]:=\{\phi\mid \varphi\}\]
where $[a_1,\dots,a_n]$ is a notation for the variables $a_1,\dots,a_n$ where $n\ge 0$, and $\{\phi\mid \varphi\}$ is a mapping of $a_1,\dots,a_n$; or we could write a sentence like ``for $a_1,\cdots,a_n$, let $[a_1,\cdots,a_n]$ be $\{\phi\mid \varphi\}$''. Note that for the case $n=0$, the notation has no variables, it is therefore only a symbol that cannot be used as a variable symbol, and we call it a \textbf{notational symbol}. Examples of common notational symbols include $\emptyset,\mathbb{N},\mathbb{Q},\mathbb{R},\mathbb{C},\aleph_0,\pi,e,0,1,2,3,4,5,6,7,8,9$ and etc.

\begin{definition}[Empty Set]$\emptyset:=\{x\mid \neg (x=x)\}$
\end{definition}
\begin{definition}[Pairing Set]$\{a,b\}:=\{x\mid x=a\lor x=b\}$
\end{definition}
\begin{definition}[Singleton]$\{a\}:=\{a,a\}$
\end{definition}
\iffalse
\begin{definition}[Finite Set]$\{a_1,\dots,a_n\}:=\{x\mid x=a_1\lor \cdots\lor x=a_n\}$
\end{definition}\fi
\begin{definition}[Tuple]$(a,b):=\{\{a\},\{a,b\}\}$
\end{definition}
\iffalse
\begin{definition}[Ordered List]$(a_1,\dots,a_n):=\left.\left(a_{1},\left(a_{2},\left(a_{3},\left(\ldots,\left(a_{n}, \emptyset\right) \ldots\right)\right)\right)\right)\right)$
\end{definition}\fi
\begin{definition}[Set Difference]$a\setminus b:=\{x\mid x\in a\land \neg (x\in b)\}$
\end{definition}
\iffalse
\begin{definition}[Cartesian Product]$a_1\times \cdots\times a_n:=\{(x_1,\dots,x_n)\mid x_1\in a_1\land\cdots\land  x_n\in a_n\}$
\end{definition}\fi
\begin{definition}[Cartesian Product]$a\times b:=\{(x,y)\mid x\in a\land y\in b\}$
\end{definition}
\begin{definition}[Union Set]$\bigcup a:=\{x\mid \exists b\, [b\in a\land x\in b]\}$
\end{definition}
\begin{definition}[Intersection Set]$\bigcap a:=\{x\mid \forall b\,[b\in a\rightarrow x\in b]\}$
\end{definition}
\begin{definition}[Union]$a\cup b:=\bigcup \{a,b\}$
\end{definition}
\begin{definition}[Intersection]$a\cap b:=\bigcap \{a,b\}$
\end{definition}
\begin{definition}[Power Set]$\mathcal{P}(a):=\{x\mid \forall y\,[y\in x\rightarrow y\in a]\}$
\end{definition}
\begin{definition}[Natural Numbers] 
\[\mathbb{N}:=\bigcap\{z\mid (\emptyset\in z)\land (\forall w\,[w\in z\rightarrow w\cup \{w\}\in z])\}\]
%\[\mathbf{N}:=\{n\mid [n=\emptyset\lor \exists k\,(n=k\cup \{k\})]\land \forall m\in n[m=\emptyset \lor \exists k\in n\,(m=k\cup \{k\})]\}\]
\end{definition}
\iffalse
\begin{definition}[Arabic Numerals]Define the notational symbols $0,1,2,3,4,5,6,7,8,9$ as
\[\begin{aligned}
0&:=\emptyset\\
1&:=\{0\}\\
2&:=\{0,1\}\\
3&:=\{0,1,2\}\\
4&:=\{0,1,2,3\}\\
5&:=\{0,1,2,3,4\}\\
6&:=\{0,1,2,3,4,5\}\\
7&:=\{0,1,2,3,4,5,6\}\\
8&:=\{0,1,2,3,4,5,6,7\}\\
9&:=\{0,1,2,3,4,5,6,7,8\}\\
\end{aligned}\]
\end{definition}\fi

An intentional definition is an abbriviation for a property by introducing a new predicate or terminology. To declare a intentional definition, we either write the string
\[\langle a_1,\dots,a_n\rangle\Longleftrightarrow P(a_1,\cdots,a_n)\]
where $a_1,\dots,a_n$ with $n\ge 1$ are variables, $\langle a_1,\dots,a_n\rangle$ is a predicate and $P(a_1,\cdots,a_n)$ is a property of $a_1,\cdots,a_n$; or we could write a sentence like ``$a_1,\dots,a_n$ is said to have blah blah relation or property if and only if (usually we don't say the `only if' part) $P(a_1,\cdots,a_n)$''. 
\env{definition}{$a\not\in b\Longleftrightarrow \neg (a\in b)$}
\env{definition}{$a\ne b\Longleftrightarrow \neg (a=b)$}
\env{definition}{$a\subseteq b\Longleftrightarrow \forall z\,[(z\in a)\rightarrow (z\in b)]$ or $a$ is a \textbf{subset} of $b$}
\env{definition}{$a\subset b\Longleftrightarrow \neg(a=b)\land\forall z\,[(z\in a)\rightarrow (z\in b)]$ or $a$ is a \textbf{proper subset} of $b$}
\env{definition}{$R$ is a \textbf{relation} between $X$ and $Y$ if $R\subseteq X\times Y$}
\env{definition}{$R$ is a relation on $X$ if $R\subseteq X\times X$}
\env{definition}{$aRb\Longleftrightarrow (a,b)\in R$}
\env{definition}{$R$ is \textbf{symmetric} if $\forall a\,\forall b\, [aRb\leftrightarrow bR a]$}
\env{definition}{$R$ is \textbf{antisymmetric} if $\forall a\,\forall b\,[(aRb\land bRa)\rightarrow (a=b)]$}
\env{definition}{$R$ is \textbf{reflexive} on $X$ if $\forall a\,[a\in X\rightarrow aRa]$}
\env{definition}{$R$ is \textbf{transitive} if $\forall a\,\forall b\,\forall c\, [(aRb\land bRc)\rightarrow aRc]$}
\env{definition}{$R$ is \textbf{total} on $X$ if $\forall a\,\forall b\,[(a\in X\land b\in X)\rightarrow (aRa\lor bRa)]$}
\env{definition}{$R$ has a \textbf{minimal element} on $X$ if $\exists a\,\forall b\, (aRb)$}
\env{definition}{$R$ has a \textbf{maximal element} on $X$ if $\exists a\,\forall b\, (bRa)$}
\env{definition}{$R$ is an \textbf{equivalence relation} if it is symmetric, reflexive, and transitive.}
\env{definition}{$R$ is a \textbf{partial order} if it is antisymmetric, reflexive, and transitive.}
\env{definition}{$R$ is a \textbf{total order} on $X$ if it is a partial order and is total on $X$.}
\env{definition}{$R$ is a \textbf{well order} on $X$ if it is a total order on $X$ and for all nonempty subsets $S$ of $X$, $R$ has a minimal element on $S$.}

\env{definition}{We say that the relation $f$ between  $X$ and $Y$ is a \textbf{function} from the \textbf{domain} $X$ to the \textbf{codomain} $Y$ if for $x\in X$ there exists a unique $y\in Y$ with $(x,y)\in f$. In other words,
\[(f:X\rightarrow Y)\Longleftrightarrow (f\subseteq X\times Y)\land(\forall x\,[x\in X\rightarrow \exists y\,[(x,y)\in f\land \forall z\,[(x,z)\in f\rightarrow z=y]]])\]}

\begin{remark}To further simplify first order language, we introduce the following abbriviations. Let $F_1,\dots,F_n,F$ be formulas, $z_1,\dots,z_n,x,y,X,Y,f$ be variables, and $t_1,\dots,t_n,t,s$ be terms.
\env{enumerate}{
	\item $F_1\land \cdots\land F_n$ abbriviates $(\cdots((F_1\land F_2)\land F_3)\cdots \land F_n)$
	\item $F_1\lor \cdots\lor F_n$ abbriviates $(\cdots((F_1\lor F_2)\lor F_3)\cdots \lor F_n)$
	\item $t_1,\cdots,t_n\in s$ abbriviates $t_1\in s\land\cdots \land t_n\in s$
	\item $(\forall z_1,\dots,z_n\in s) F$ abbriviates $\forall z_1\,\cdots\forall z_n\,([z_1\in s\land \cdots\land z_n\in s]\rightarrow F)$
	\item $(\exists z_1,\dots,z_n\in s) F$ abbriviates $\exists z_1\,\cdots\exists z_n\,([z_1\in s\land\cdots\land z_n\in s]\land F)$
	\item $\exists! x\, F$ abbriviates $\exists x\, [F\land ((\exists y\, F)\rightarrow (x=y))]$
	\item $\{t\in s\mid F\}$ abbbriviates $\{t\mid (t\in s)\land F\}$
	\item $(\forall f:X\rightarrow Y)\,F$ abbriviates $\forall f\,((f:X\rightarrow Y)\rightarrow F)$
	\item $(\exists f:X\rightarrow Y)\,F$ abbriviates $\exists f\,((f:X\rightarrow Y)\land F)$
	\item $\{f:X\rightarrow Y\mid F\}$ abbriviates $\{f\mid (f:X\rightarrow Y)\land F\}$
}
\end{remark}
\newpage
\env{exercise}{Consider the definitions from Definition \ref{def:start} to Definition \ref{def:end}. Which ones are intentional definitions? Which ones are extensional definitions?}
\env{definition}{\label{def:start}$f(x):=\bigcup\{y\mid(x,y)\in f\}$}
\env{definition}{Let the \textbf{image} of $A$ under $f$ be $f[A]:=\{f(x)\mid x\in A\}$}
\env{definition}{Let the \textbf{preimage} of $B$ under $f$ be $f^{-1}[B]:=\{x\mid f(x)\in B\}$}
\env{definition}{$\im f:=\{y\mid \exists x\,f(x)=y\}$}
\env{definition}{$f$ is \textbf{one-to-one} on $S$ if $\forall a,b\in S\,[(f(a)=f(b))\rightarrow (a=b)]$}
\env{definition}{$f$ is \textbf{onto} $S$ if $\forall y\in S\,\exists x\,(f(x)=y)$}
\env{definition}{A function is \textbf{injective} if it is one-to-one on its domain}
\env{definition}{A function is \textbf{surjective} if it is onto its codomain}
\env{definition}{A function is \textbf{bijective} if it is injective and surjective}
\env{definition}{$f^{-1}:=\{(y,x)\mid (x,y)\in f\}$}
\env{definition}{$f\circ g:=\{(x,z)\mid \exists y\,[(x,y)\in f\land (y,z)\in g]\}$}
\env{definition}{Define the \textbf{equivalence class} of $x$ under $R$ as $[x]_R:=\{y\mid xRy\}$}
\env{definition}{Define $\preceq:=\{(a,b)\mid (\exists f:a\rightarrow  b)\,[f\textrm{\ is\ injective}]\}$}
\env{definition}{Define $\succeq:=\{(a,b)\mid (\exists f:a\rightarrow  b)\,[f\textrm{\ is\ surjective}]\}$}
\env{definition}{Define $\simeq:=\{(a,b)\mid (\exists f:a\rightarrow  b)\,[f\textrm{\ is\ bijective}]\}$}
\env{definition}{Define \textbf{cardinality} of $x$ as $\card(x):=[x]_{\simeq}$}
\env{definition}{$y$ is a \textbf{cardinal number} if $\exists x\,(y=\card(x))$}
\env{definition}{Define $\le:=\{(a,b)\mid \forall x\in a\,\forall y\in b\,(x\preceq y)\}$}
\env{definition}{Define $\ge:=\{(a,b)\mid \forall x\in a\,\forall y\in b\,(x\succeq y)\}$}
\env{definition}{Define $<:=\{(a,b)\mid a\le b\land a\ne b\}$}
\env{definition}{Define $>:=\{(a,b)\mid a\ge b\land a\ne b\}$}
\env{definition}{$\aleph_0:=\card(\mathbb{N})$}
\env{definition}{$a$ is \textbf{finite} if $\card(a)< \aleph_0$}
\env{definition}{$a$ is \textbf{countable} if $\card(a)\le \aleph_0$}
\env{definition}{$a$ is \textbf{countably infinite} if $\card(a)=\aleph_0$}
\env{definition}{$a$ is \textbf{uncountable} if $\card(a)>\aleph_0$}
\env{definition}{$a$ is \textbf{inductive} if $\emptyset\in a\land\forall x\,[x\in a\rightarrow (x\cup \{x\})\in a]$}
\env{definition}{$a$ is \textbf{set-transitive} if $\forall x\,\forall y\,[(x\in a\land y\in x)\rightarrow y\in a]$}
\env{definition}{$a$ is an \textbf{ordinal number} if it is set-transitive and totally ordered by $\subseteq$}
\env{definition}{$*$ is an \textbf{operation} on $X$ if $*:X\times X\rightarrow X$}
\env{definition}{\label{def:end}$a*b:=*((a,b))$}

\newpage

\section{Zermelo–Fraenkel Axioms}
Bear in mind that so far we have not assigned truthness to formulas. To introduce truthness into first order language, we first have to state a list of \textbf{axioms}. A list of axioms is a list of independent and non-contradictory propositions that are taken to be true for granted. Any proposition inferred from the list of axioms is considered true, and any proposition whose negation is true is considered false. Note that a proposition can be neither true or false, but cannot be both true and false. In fact, the logician Kurt G\"odel proved that almost all meaningful formal axiomatic system contains a proposition that is neither true or false, which is known as G\"odel's incompleteness theorem.

 A string of the form $F_1,\dots,F_n\vDash G$ where $F_1,\dots,F_n,G$ are formulas is called an \textbf{argument}. If $n=0$, that is, we assume no premise other than ZFC, we write $\mathrm{ZFC}\vDash G$. The formulas $F_1,\dots,F_n$ are called the \textbf{premises}, and $G$ is called the \textbf{conclusion}. An ordered finite list of arguments that follow the \textbf{Basic Validity Rules} and ends with $F_1,\dots,F_n\vDash G$ is called a \textbf{proof} for the argument $F_1,\dots,F_n\vDash G$. Arguments with proofs are called \textbf{theorems}; theorems that are meant to be used in other proofs are called \textbf{lemmas}; theorems that can be proved easily by using another theorem are called \textbf{corollaries}; and long-standing unproved arguments are called \textbf{conjectures}. 

\begin{axiom}[Axiom of Empty Set]$\exists w\, \forall x\,(x\not\in w)$
\end{axiom}
\begin{axiom}[Axiom of Extensionality]$\forall x\,\forall y\,[(x=y)\leftrightarrow \forall z\,[(z\in x)\leftrightarrow (z\in y)]]$
\end{axiom}
\begin{axiom}[Axiom of Pairing]$\forall a \,\forall b \,\exists w \,\forall x\,[(x \in w) \leftrightarrow (x=a \lor x=b)]$
\end{axiom}
\begin{axiom}[Axiom of Union]$\forall a\,\exists w\,\forall x\,(x\in w\leftrightarrow \exists b\, [b\in a\land x\in b])$
\end{axiom}
\begin{axiom}[Axiom of Power Set]$\forall a \,\exists w\, \forall x\,[x \in w \leftrightarrow x\subseteq a]$
\end{axiom}
\begin{axiom}[Axiom of Infinity]$\exists w\,(\emptyset\in w\land\forall x\in w\,[(x\cup \{x\})\in w])$
\end{axiom}
\begin{axiom}[Axiom of Regularity]$\forall x\,(x\ne \emptyset\rightarrow \exists y\in x\,(y\cap x=\emptyset))$
\end{axiom}
The axiom of empty set ensures the existence of $\emptyset$ as we defined in the last section; the axioms of pairing, union and power set ensures the existence of pairs, unions and power sets; the axiom of extensionality dictates that two sets are equal if and only if they have the same elements; the axiom of infinity guarentees the existence of an \textbf{inductive set}; and the axiom of regularity excludes the possibility of a set that contains itself. The last two of the ZF ``axioms'', however, cannot be formulated as first order propositions, as they involve in formulas as variables. The only way to formulate them as propositions is to introduce second order language, which  is not a topic of our discussion. Therefore, we will instead write ``for all such and such formulas'' as an informal substitude. Rigorously speaking, they are not two single axioms in first order language, but ``a lot of them''. We call them \textbf{axiom schemas}.
\begin{axiom}[Axiom Schema of Specification]For all formulas $\varphi$ about $x, a,w_{1}, \ldots, w_{n}$,
\[\forall w_{1}\, \cdots\forall w_{n}\, \forall a\, \exists b \,\forall x\,\left(x \in b \leftrightarrow\left[x \in a \wedge \varphi\right]\right)\]
is a ZF axiom. We allow $n=0$, in which case there is no variable $w_i$.
\end{axiom}
\begin{axiom}[Axiom Schema of Replacement]For all formulas $\varphi$ about $a,x, y,w_{1}, \ldots, w_{n}$ such that $b$ is unused in $\varphi$,
\[\forall w_{1}, \ldots,\forall w_{n}\, \forall a\,\left(\left[\forall x \in a\, \exists! y \,\varphi\right] \rightarrow \exists b\, \forall y\,\left[y \in b \leftrightarrow \exists x \in a\, \varphi\right]\right)\]
is a ZF axiom. We allow $n=0$, in which case there is no variable $w_i$.
\end{axiom}\newpage
The axiom schema of specification essentially says that every definable subclass of a set is a set, and the axiom of replacement says that the image of a function is a set, which we will discuss in Section \ref{sec:cardinal}. There is one more axiom, the axiom of choice, that we have not talked about. This axiom involves in much more technicality, thus we will talk about it in Section \ref{sec:axoimofchoice}. Now, let $F_1,\dots,F_n,F,G,H,K$ be formulas and $x,y$ be distinct variables and $s,t,h$ be terms, then the following are basic validity rules, where ``iff'' is an abbriviation for ``if and only if''
\env{enumerate}{
	\item If $F$ is a ZFC axiom, then $F_1,\dots,F_n\vDash F$.
	\item If $1\le i\le n$ then $F_1,\dots,F_n\vDash F_i$.
	\item If $G\equiv H$ then $F_1,\dots,F_n\vDash G$ iff $F_1,\dots,F_n\vDash H$.
	\item If $G\equiv H$ then $F_1,\dots,F_n,G\vDash K$ iff $F_1,\dots,F_n,H\vDash K$.
	\item If $F_1,\dots,F_n\vDash G$ then $F_{n_1},\dots,F_{n_k}\vDash G$ for $1\le n_1<\cdots<n_k\le n$.
	\item If $F_1,\dots,F_n\vDash G$ and $F_1,\dots,F_n,G\vDash H$ then $F_1,\dots,F_n\vDash H$.
	\item If $F_1,\dots,F_n,G\vDash H$ and $F_1,\dots,F_n,\neg G\vDash H$ then $F_1,\dots,F_n\vDash H$.
	\item If $F_1,\dots,F_n,\neg G\vDash H$ and $F_1,\dots,F_n,\neg G\vDash \neg H$ then $F_1,\dots,F_n\vDash \neg G$.
	\item $F_1,\dots,F_n\vDash G\land H$ iff $F_1,\dots,F_n\vDash G$ and $F_1,\dots,F_n\vDash H$.
	\item $F_1,\dots,F_n,G,H\vDash K$ iff $F_1,\dots, F_n,G\land H\vDash K$.
	\item $F_1,\dots,F_n\vDash F\lor G$ iff $F_1,\dots,F_n,\neg G\vDash F$.
	\item If $F_1,\dots,F_n,F\vDash H$ and $F_1,\dots,F_n,G\vDash H$ then $F_1,\dots,F_n,F\lor G\vDash H$.
	\item If $F_1,\dots,F_n\vDash G$ then $F_1,\dots,F_n\vDash G\lor H$.
	\item If $F_1,\dots,F_n\vDash G\lor H$ and $F_1,\dots,F_n\vDash \neg G$ then $F_1,\dots,F_n\vDash H$.
	\item $F_1,\dots,F_n\vDash G\rightarrow H$ iff $F_1,\dots,F_n,G\vDash H$.
	\item $F_1,\dots,F_n\vDash G\rightarrow H$ iff $F_1,\dots,F_n,\neg H\vDash \neg G$.
	\item If $F_1,\dots,F_n,\neg F\vDash H$ and $F_1,\dots,F_n,G\vDash H\rightarrow H$ then $F_1,\dots,F_n,F\rightarrow G\vDash H$.
	\item If $F_1,\dots,F_n\vDash \neg G$ then $F_1,\dots,F_n\vDash G\rightarrow H$.
	\item If $F_1,\dots,F_n\vDash H$ then $F_1,\dots,F_n\vDash G\rightarrow H$.
	\item If $F_1,\dots,F_n\vDash G\rightarrow H$ and $F_1,\dots,F_n\vDash G$ then $F_1,\dots,F_n\vDash H$.
	\item If $F_1,\dots,F_n\vDash G\rightarrow H$ and $F_1,\dots,F_n\vDash \neg H$ then $F_1,\dots,F_n\vDash \neg G$.
	\item $F_1,\dots,F_n\vDash G\leftrightarrow H$ iff $F_1,\dots,F_n\vDash H\rightarrow G$ and $F_1,\dots,F_n\vDash G\rightarrow H$.
	\item If $F_1,\dots,F_n,G,H\vDash K$ and $F_1,\dots,F_n,\neg G,\neg H\vDash K$ then $F_1,\dots,F_n,G\leftrightarrow H\vDash K$.
	\item If $F_1,\dots,F_n\vDash G$ and $F_1,\dots,F_n\vDash H$ then $F_1,\dots,F_n\vDash G\leftrightarrow H$.
	\item If $F_1,\dots,F_n\vDash \neg G$ and $F_1,\dots,F_n\vDash \neg H$ then $F_1,\dots,F_n\vDash G\leftrightarrow H$.
	\item If $F_1,\dots,F_n\vDash G\leftrightarrow H$ and $F_1,\dots,F_n\vDash G$ then $F_1,\dots,F_n\vDash H$.
	\item If $F_1,\dots,F_n\vDash G\leftrightarrow H$ and $F_1,\dots,F_n\vDash \neg G$ then $F_1,\dots,F_n\vDash \neg H$.
	\item If $\vDash G$ then $F_1,\dots,F_n\vDash G$.
	\item If $\vDash G$ then $F_1,\dots,F_n,G\vDash H$ iff $F_1,\dots,F_n\vDash H$.
	\item $F_1,\dots,F_n\vDash t=t$.
	\item If $F_1,\dots,F_n\vDash s=t$ and $F_1,\dots,F_n\vDash t=h$ then $F_1,\dots,F_n\vDash s=h$.
	\item If $F_1,\dots,F_n\vDash s=t$ then $F_1,\dots,F_n\vDash [G][x\mapsto s]$ iff $F_1,\dots,F_n\vDash [G][x\mapsto t]$.
	\item If $F_1,\dots,F_n\vDash \forall x\,G$ then $F_1,\dots,F_n\vDash [G][x\mapsto t]$.
	\item If $F_1,\dots,F_n\vDash [G][x\mapsto t]$ then $F_1,\dots,F_n\vDash \exists x\, G$.
	\item If $F_1,\dots,F_n,[G][x\mapsto t]\vDash H$ then $F_1,\dots,F_n,\forall x\, G\vDash H$.
	\item If $F_1,\dots,F_n,\exists x\,G\vDash H$ then $F_1,\dots,F_n, [G][x\mapsto t]\vDash H$.
	\item If $F_1,\dots,F_n\vDash [G][x\mapsto y]$, $x$ is free in $G$, and $y$ is not free in $F_1,\dots,F_n, G$ then \[F_1,\dots,F_n\vDash\forall x\, G\]
	\item If $F_1,\dots,F_n,[G][x\mapsto y]\vDash H$ where $y$ is not free in $F_1,\dots,F_n,G,H$, then 
	\[F_1,\dots,F_n,\exists x\,G\vDash H\]
	\item If $1\le i<j\le n$ and $F_1,\dots,F_n\vDash G$, then
	\[F_1,\dots,F_{i-1},F_{j},F_{i+1},\dots,F_{j-1},F_{i},F_{j+1},\dots,F_n\vDash G\]
}
If we declare that $\mathcal{S}:=\{\phi\mid \varphi\}$ where $\{\phi\mid \varphi\}$ ranges over $a_1,\dots,a_n$, and we proved that 
\[\mathrm{ZFC}\vDash \exists x\,\forall y \,[y\in x\leftrightarrow \,\exists a_1\,\cdots \exists a_n\,(y=\phi\land \varphi)]\]
then we say that $\mathcal{S}$ is a set.
Natural Numbers, Integers and induction
\newpage
\section{Rational Numbers, Real Numbers and Complex Numbers}
Binomial Coefficient
Archimedian Property
Denseness
Cantor Set
\section{Ordinal Numbers and Cardinal Numbers}
\label{sec:cardinal}
Transfinite induction
Continuum Hypothesis
Schurodiner Burnsiein
Grothendieck Universe
Von Neumann Universe

\section{Axiom of Choice and Zorn's Lemma}
\label{sec:axoimofchoice}
Zorn Lemma
Tukey Lemma
Hausdorff Maximal principle
Well ordering theorem
Banach-Tarski paradox

\section{Algorithms and Turing Computability}
Boolean satisfiabililty problem
Entscheidungsproblem
\section{Generalizations of ZFC Set Theory}





\chapter{Number Theory and Beyond}
\section{Motivation}
\newpage
\section{Rings}
We start by introducing the idea of \textbf{algebraic structures}. In mathematics, algebraic structures are sets with operations that follows a list of properties (sometimes called axioms, but not to be confused with the logical ZFC axioms) that originated from our intuitions of elementary and common mathematical objects. 
\env{definition}{
	Define an operation ($*$) on $G$ such that
	\env{enumerate}{
		\item exists \textbf{identity} $e\in G$ such that for $a\in G$, $a*e=e*a=a$
		\item for $a,b,c\in G$, $(a*b)*c=a*(b*c)$, 
		\item for $a\in g$, exists \textbf{inverse} $a^{-1}\in G$ such that $a*a^{-1}=a^{-1}*a=e$,
	}
	then $G$ is called a \textbf{group} under the operator ($*$). Further, if a group $G$ satisfies the property
	\env{enumerate}{
		\setcounter{enumi}{3}
		\item for $a,b\in G$, $a*b=b*a$,
	}
	then $G$ is an \textbf{abelian group}. 
}
\env{definition}{
	Define two operations \textbf{addition} ($+$) and \textbf{multiplication} ($\cdot$) on $R$ such that
	\env{enumerate}{
		\item $R$ is an abelian group under ($+$) with identity $0$ and inverse $-a$ for $a\in R$.
		\item for $a,b,c\in R$, $a\cdot (b\cdot c)=(a\cdot b)\cdot c$,
		\item for $a,b,c\in R$, $a\cdot (b+c)=(a\cdot b)+(a\cdot c)$ and $(b+c)\cdot a=(b\cdot a)+(c\cdot a)$,
		\item exists \textbf{multiplicative identity} $1\in R$ such that for $a\in R$, $1\cdot a=a\cdot 1=a$,
	}
	then $R$ is called a \textbf{ring} under the addition ($+$) and the multiplication ($\cdot$). Further, consider 
	\env{enumerate}{
		\setcounter{enumi}{7}
		\item for $a,b\in R$, $a\cdot b=b\cdot a$,
		\item for $a\in R$, exists \textbf{multiplicative inverse} $a^{-1}\in X$ such that $aa^{-1}=a^{-1}a=1$
	}
	If a ring $R$ satisfies property 8 then it is a \textbf{commutative ring}. If a ring $R$ satisfies property 9 then it is a \textbf{division ring}. A commutative division ring is called a \textbf{field}. A \textbf{pseudo-ring} is a set $R$ with two operations addition ($+$) and multiplication ($\cdot$) such that properties 1,2,3 are satisfied but not necessaily property 4. 
}
\begin{proposition}
If $G$ is a group then 
\env{enumerate}{
	\item $e\in G$ is unique
	\item for $g\in G$, $g^{-1}$ is unique
	\item for $g\in G$, $(g^{-1})^{-1}=g$
	\item for $g,\dots,g_n\in G$, $(g_1\cdots g_n)^{-1}=g_n^{-1}\cdots g_1^{-1}$
}
\end{proposition}
\begin{proposition}
If $R$ is a ring and $a,b,c\in R$, then
\env{enumerate}{
	\item $a0=0a=0$
	\item $a(-b)=(-a)b=-(ab)$
	\item $(-a)(-b)=ab$
	\item $(-1)a=-a$
	\item $(-1)(-1)=1$
	\item $a(b-c)=ab-ac$ and $(b-c)a=ba-ca$
	\item $0,1$ and $-a$ are unique
	\item $a^{-1}$ is unique if it exists
	\item for $a_1,\dots,a_n\in R$, $(a_1\cdots a_n)^{-1}=a_n^{-1}\cdots a_1^{-1}$ if all inverses in this identity exists
	\item if $a,b\in R$ and $n\in \mathbf{N}$, then
	\[(a+b)^n=\binom{n}{0}a^n+\binom{n}{1}a^{n-1}b+\binom{n}{2}a^{n-2}b+\cdots+\binom{n}{n}b^n\]
	where $\binom{n}{k}$ is the binomial coefficients.
	\item for $a,b\in R$ and $n\in\mathbf{N}$
	\[a^n-b^n=(a-b)(a^{n-1}+a^{n-2}b+a^{n-3}b^2+\cdots +ab^{n-2}+b^{n-1})\]
	and if $n$ is odd
	\[a^n+b^n=(a+b)(a^{n-1}-a^{n-2}b+a^{n-3}b^2-\cdots \pm ab^{n-2}\mp b^{n-1})\]
}
\end{proposition}
\begin{example}Here are some examples of groups
\env{enumerate}{
	\item $\mathbf{Z},\mathbf{Q},\mathbf{R},\mathbf{C}$ are groups under addition
	\item The \textbf{cyclic group} of degree $n$, $\mathrm{C}_n=\{1,g,\dots,g^{n-1}\}$ where $g^n=1$.
	\item The \textbf{dihedral group} of degree $n$, $\mathrm{D}_n=\{1,r,\dots,r^{n-1},s,sr,\dots,sr^{n-1}\}$ where $s^2=r^n=1$.
	\item The \textbf{permutation group} of degree $n$ is 
	\[\mathrm{S}_n=\{f:\mathbf{Z}^{+}\rightarrow \mathbf{Z}^{+}\mid f \textrm{\ is\ bijective\ and\ }f(m)=m\textrm{\ for\ }m>n\}\]
	where the operation is function composition.
	\item The \textbf{quaternion group}, $\mathrm{Q}_8=\{1,-1,i,-i,j,-j,k,-k\}$ where $(-1)^2=1$ and \[i^2=j^2=k^2=ijk=-1\]
	\item If $R$ is a ring, define $R^\times=\{r\in R\mid \exists s\in R\,( rs=sr=1)\}$, then $R^\times$ is a group under the multiplication operation of $R$, which we call the \textbf{multiplicative group of} $R$.
	\item If $G_1,\dots,G_n$ are groups, then $G_1\times \cdots\times G_n$ is a group under the operation 
	\[(a_1,\dots,a_n)(b_1,\dots,b_n)=(a_1b_1,\dots,a_nb_n)\] 
	for $a_1,b_1\in G,\dots,a_n,b_n\in G_N$, which we call the \textbf{product group} of $G_1,\dots,G_n$.
}
\end{example}
\begin{example}Here are some examples of rings
\env{enumerate}{
	\item $\mathbf{Z},\mathbf{Q},\mathbf{R},\mathbf{C}$ are rings under the usual operations, and $\mathbf{Q},\mathbf{R},\mathbf{C}$ are also fields.
	\item If $R_1,\dots,R_n$ are rings then $R_1\times \cdots\times R_n$ is the called the \textbf{product ring} of $R_1,\dots,R_n$ where
	\[\begin{aligned}
	(a_1,\dots,a_n)+(b_1,\dots,b_n)&=(a_1+b_1,\dots,a_n+b_n)\\
	(a_1,\dots,a_n)(b_1,\dots,b_n)&=(a_1b_1,\dots,a_nb_n)\\
	\end{aligned}\]
	for $a_1,b_1\in R_1, \dots, a_n,b_n\in R_n$. 
	\item If $G$ is a group and $R$ is a ring, then the \textbf{group ring} of $R$ over $G$
	\[
	R[G]=\{f:G\rightarrow R\mid |f^{-1}(R\setminus\{0\})|<\aleph_0 \}\]
	is a ring under the operations 
	\[\begin{aligned}
	(f_1+f_2)(g)&=f_1(g)+f_2(g)\\
	(f_1f_2)(g)&=\sum_{h\in G}f_1(h)f_2(h^{-1}g)
	\end{aligned}\]
	for $g\in G$ and $f_1,f_2\in R[G]$. We typically write an element $f\in R[G]$ as the formal polynomial 
	\[a_1g_1+\cdots +a_ng_n\]
	where $\{g_1,\dots,g_n\}=f^{-1}(R\setminus\{0\})$ and $a_i=f(g_i)$ for $1\le i\le n$. Define the group $\langle x\rangle=\{x^n\mid n\in\mathbf{Z}\}$, then we abbriviate $R[\langle x\rangle]$ as $R[x]$, which we call the \textbf{polynomial ring} of $R$.
	\item Let $R$ be a ring, then $R[[x]]=\{f\mid f:\mathbf{N}\rightarrow R\}$ is called the \textbf{power series ring} of $R$ under 
	\[\begin{aligned}
	(f_1+f_2)(n)&=f_1(n)+f_2(n)\\
	(f_1f_2)(n)&=\sum_{i=0}^nf_1(i)f_2(n-i)
	\end{aligned}\]
	for $n\in \mathbf{N}$. We typically write an element $f\in R[[x]]$ as the formal power series
	\[a_0+a_1x+a_2x^2+\cdots\]
	where $a_i=f(i)$ for $i\ge 0$.
	\item If $R$ is a ring then $\mathcal{M}_n(R)=\{f\mid f:\{1,\dots,n\}^2\rightarrow R\}$ is called the \textbf{matrix ring} under 
	\[\begin{aligned}
	(f_1+f_2)(i,j)&=f_1(i,j)+f_2(i,j)\\
	(f_1f_2)(i,j)&=\sum_{k=1}^n f_1(i,k)f_2(k,j)
	\end{aligned}\]
	for $1\le i,j\le n$ and $f_1,f_2\in \mathcal{M}_n(R)$. We typically write an element $f\in \mathcal{M}_n(R)$ as 
	\[\begin{pmatrix}
	a_{1,1}&a_{1,2}&\cdots& a_{1,n}\\
	a_{2,1}&a_{2,2}&\cdots& a_{2,n}\\
	\vdots &\vdots&\ddots&\vdots\\
	a_{n,1}&a_{n,2}&\cdots &a_{n,n}
	\end{pmatrix}\]
	where $a_{i,j}=f(i,j)$ for $1\le i,j\le n$.
}
\end{example}
\section{Irreducibles and Primes}
\section{ED, PID, and UFD}
\chapter{Finite Dimensional Vector Spaces}
\section{Motivation}
\chapter{Calculus and Beyond}
\section{Motivation}
\newpage
\section{Metric Space and Metric Topology}
Herschfeld's Convergence Theorem
\chapter{Measure and Integration}
\section{Motivation}


\end{document}
